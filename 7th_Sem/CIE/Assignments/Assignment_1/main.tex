\documentclass[a4paper]{article} 
\input{style/head.tex}

\newcommand{\yourname}{Manas Chebrolu}
\newcommand{\yoursrn}{PES1UG20CS111}
\newcommand{\coursecode}{UE20EC400F}
\newcommand{\assignmentnumber}{1}

\begin{document}
\fancyhead[C]{}
\hrule \medskip
\begin{minipage}{0.295\textwidth} 
\raggedright
\footnotesize
\yourname \hfill\\ 
\yoursrn \hfill\\ 
\coursecode
\end{minipage}
\begin{minipage}{0.4\textwidth} 
\centering 
\large 
CIE L1 Assignment \assignmentnumber\\ 
\normalsize 
India Startup Ecosystem\\ 
\end{minipage}
\begin{minipage}{0.295\textwidth} 
\raggedleft
\hfill{06/09/23}
\end{minipage}
\medskip\hrule 
\bigskip


% Q1
\section{(Official / Govt of India) definition of a Tech Startup}

\large{For an entity to be recognized as a startup, it shall abide by the following conditions:
\begin{itemize}
    \item An entity that is working towards innovation, development, deployment and commercialization of new products, processes or services driven by technology or intellectual property.
    \item The entity should originate from India.
    \item The period of existence and operation of the company should not exceeded 10 years from the date of establishment.
    \item It should be an original entity, i.e. it can't be formed due to a split up or a reconstruction of another business. \cite{startupindia}
    \item Should have an annual turnover not exceeding Rs. 100 crore for any of the financial years since incorporation. \cite{startupindia}
\end{itemize}
}

\rule{\textwidth}{0.4pt}

% Q2
\section{What is a DeepTech Startup and how is it different from a 'regular' Tech Startup?}

\large{DeepTech start-ups are active tech start-ups that create, deploy or utilize advanced tech in their products and services.}

\large {They utilize advanced technologies such as AI/ML, IOT, Blockchain, Big Data \& Analytics, AR/VR etc.
They are:
\begin{itemize}
    \item Complex: Have complicated technical solutions that have the potential to redefine or create new markets.
    \item An intersection of technologies: That is they use multiple technologies to solve any problem.
\end{itemize}

They are different from regular tech startups that DeepTech start-ups require more technical expertise
}

\rule{\textwidth}{0.4pt}

% Q3
\section{What are the characteristics of an "Inventive DeepTech" startup?}

\large{From the pool of DeepTech start-ups, there is an inventive set of start-ups that are creating new products or solutions that are generally backed by fundamental research.
The main characteristics include:
\begin{itemize}
    \item Development of new IP involving scientific advances.
    \item Create new ways of computing, communicating, manufacturing resulting in a meaningful engineering innovation.
\end{itemize}
}

\rule{\textwidth}{0.4pt}

\newpage

% Q4
\section{Read the section on "DeepTech Ecosystem Landscape" and answer the following:}

\subsection{Number of DeepTech startups and growth rate}

\large{Considering the time period of the last 10 years, the number of DeepTech start-ups come upto 3000+ growing at the rate of 53\%CAGR over the same time period.
}

\subsection{Top 2 sectors for top adopters of DeepTech startups}

\large {The top 2 sectors for top adopters of DeepTech startups are in the \textit{Enterprise Tech} and the \textit{BFSI Sector}
}

\subsection{A set of inventive DeepTech startups are creating solutions and value based on Intellectual Property (IP). List any 5 IP focus areas.}

\large{The Key Intellectual Property (IP) focus areas are:
\begin{enumerate}
    \item Video Analytics \& Data Analytics Platforms
    \item Customer Engagement Platforms
    \item Healthcare Diagnostics
    \item Autonomous Driving
    \item Automation Robots
\end{enumerate}
}

\rule{\textwidth}{0.4pt}

% Q5
\section{Read the section on "Tech Stack Trends"; apart from Artificial Intelligence, what are the 3 other focus areas?}

\large{Apart from Artificial Intelligence, the 3 other focus areas are:
\begin{enumerate}
    \item Big Data \& Analytics
    \item Internet of Things
    \item Blockchain
\end{enumerate}
}

\rule{\textwidth}{0.4pt}

% Q6
\section{What are the key characteristics of the Indian DeepTech startups (see Pg 17 in PDF)? Based on this section, answer the below:}

\subsection{Pick any 2 of the characteristics mentioned here and answer the following:
\\What is the significance of this being a key enabler of the Indian DeepTech startup?}

\large{
\begin{itemize}
    \item Higher Seed and Early Stage Funding: DeepTech start-ups require greater funding than regular start-ups are the requirement and skill required is greater in terms of the technical field. As mentioned, the share of seed and early stage funding is $\sim$46\% of total funding raised by DeepTech start-ups while the same is $\sim$22\% for overall tech start-ups.
    \item Enterprise Customer Focus: More than 75\% of the DeepTech start-ups are building solutions for enterprise i.e. B2B problems, compared to only 40\% of overall tech start-ups. This could be one of the key enablers, as many issues or problems faced by the B2B sector could find solutions here.
\end{itemize}
}

\subsection{Your opinion - why do you think there are fewer unicorns in the DeepTech startup space?}

\large{DeepTech start-ups begin with innovation in mind, in that sense, any start-up begins their journey with innovation in mind, but DeepTech start-up has a higher barrier of the skill requirement and knowledgeable members.
\\At least 15\% of workforce for majority of the DeepTech start-ups are skilled in the deep technologies, and acquiring this talent is a challenge; And even after acquiring the talent, it is not certain that whatever they are trying to achieve is feasible or achievable.
\\Innovation and breakthroughs are not achieved without time, effort and skill, and hence the chances for a DeepTech startup to turn unicorn are less.
}

\rule{\textwidth}{0.4pt}

% Q7
\section{Pick any two of the 8 areas listed (Pg 12 in the PDF "There are 3000+ startups working across mature DeepTech Technologies").
Pick one startup in each area and research about these 2 startups and answer the following questions for each of the startup:}

\large{Companies:
\begin{itemize}
    \item IoT - NemoCare \cite{nemocare}
    \item Drones - Tech Eagle \cite{techeagle}
\end{itemize}
}

\subsection{What is the key problem the startup is trying to solve?}

\large{
\begin{itemize}
\item \textbf{NemoCare}: NemoCare is a company in the IoT field which aims to end all neonatal and maternal deaths in the developing world by building innovative, affordable, accessible, highly accurate monitoring solutions.
\item \textbf{Tech Eagle}: TE is a company in the Drones field which aims to solve the key problem of a network of transport in the air over land for healthcare and logistics accessibility in rural and urban areas.
\end{itemize}
}

\subsection{What is the differentiation the startup is trying to provide through its solution?}

\large{
\begin{itemize}
    \item \textbf{NemoCare}: Their flagship product \textit{NemoCare Raksha} is a wearable on the newborn that will continuously monitor for necessary vital parameters used to detect critical distress conditions. An integrated diagnostic tool that connects wirelessly to a central platform, which ensures the nurse can monitor all the babies simultaneously and alert when a distress condition is detected. The usage of the product requires no prerequisite training skill and can also be used at the patient's home.
    \item \textbf{Tech Eagle}: They were Asia's 1st cold chain vaccine delivery via Drone with the Govt. of Telangana. True to their aim and goal of making transport to rural and urban areas possible for healtcare and logistics. Furthermore, they expanded to partner up with Zomato for food delivery as well.
\end{itemize}
}

\subsection{If the startup is successful, what impact(financial, societal, etc) will it have in the marketplace \&/or economy?}

\large{
\begin{itemize}
    \item \textbf{NemoCare}: If the startup is successful, it would have a major societal impact as premature death and maternal death during birth are huge contributors to the global death count, and finding a solution to this would increase the demand for the product multi-fold in the marketplace. Prevention is better than cure, and here it would mean a life to prevent the death of a newborn. Eventually the product could be manufactured at a cheaper rate and sold for more affordable pricing causing its share in the market to rise and bringing in the necessary revenue and need for further innovation.
    \item \textbf{Tech Eagle}: If this startup is successful, it could revolutionize the transport sector for small items. Majority of the transport of good is done over land or sea, which is either expensive or time consuming. Air transport could be made cheap and quick, even recent innovation in bringing back the airships have begun as they have massive potential in reducing the time to transport goods. Air travel is more free and not restricted (while still having to follow certain air routes) to the ground routes and linear motion. The transport marketplace would try to rely on air travel for transport of goods to remote places that are normally not easily accessible by land; and at the same time increase the demand and profits in the marketplace.
\end{itemize}
}

\subsection{If you were an angel investor or VC, would you invest in these 2 startups? Why?}

\large{
\begin{itemize}
    \item \textbf{NemoCare}: If I were an angel investor, I would definitely think about investing in this company, provided it already has some backing from other VC's, if my contribution could help advance their research in preventing premature death in any form or way, then I would be glad to have put the investment to use. But, thinking of the business perspective, healthcare will always be in demand no matter what age, and hence having the ability to contribute to prevention of premature death and a stock in a large potential market cap.
    \item \textbf{Tech Eagle}: Another innovative company that will revolutionize transport in the upcoming years, and I would definitely like to be part of their journey if I were a VC by investing into the company. I see huge potential for their profits and the pick up of their product. Their existing solutions have already been utilized for healthcare product transport to remote areas along with faster food deliveries. By also reducing the carbon footprint by reducing carbon emissions and promoting green energy, they have found a sweet spot for transport of commercial goods. The cause and the effort both exist in this product and it definitely has the potential to make huge profits in the business side.
\end{itemize}
}

\rule{\textwidth}{0.4pt}

\newpage

% Q8
\section{Based on the 8 areas* (Pg \#12) pick any 2 tech and answer the below for each of the tech areas
\\\textit{*AI, Blockchain, IoT, Big Data \& Analytics, AR/VR, Robotics, Drones,3D Printing}}

\large{Choice of areas:
\begin{itemize}
    \item AI/ML
    \item AR/VR
\end{itemize}
}

\subsection{Share \textit{your} understanding (what/why/application/etc) of the deep-tech in $\sim$100 words}

\large{
\begin{center}
    \textbf{\underline{AI/ML}}
\end{center}

AI / ML is the current buzzword, the trend and it speaks for itself; Artificial Intelligence, is the replication of human intelligence in machines for any particular or a range of tasks. Machine Learning would be the process in which a machine would learn about any particular field, just as how humans have to learn.
\\Why would one need AI, AI can replace humans in the repetitive tasks that require more effort by humans, and machines do not get tired as quickly as humans do, require less rest but do require constant maintenance.
\\AI is an umbrella term for many other fields that hide underneath it and it can be used almost everywhere. A recent breakthrough in AI would be ChatGPT, which is a Large Language Model, which basically means that it has been trained on a large corpus of filtered text data. It can answer like a human would, with strikingly similar accuracy and speed.
\\In conclusion, it is the domain to be working on providing major breakthroughs and innovative solutions for problems faced by humanity.

\begin{center}
    \textbf{\underline{AR/VR}}
\end{center}

Augmented Reality / Virtual Reality or more commonly known as AR/VR, is the domain in which reality can be bent. VR is achieving something that can't be achieved in reality, hence the term virtual, it enables the user to be engrossed in a virtual space while still remaining at the same location physically. This allows people to be connected without having to meet physically, which can have its ups and downs.
\\AR on the other hand is just adding to the existing reality. Instead of taking you to another reality, they bring the other to your reality, famous products have already been built to support the AR domain such as the google lens, but it hasn't seen its potential yet
\\AR/VR can have massive application potential in gaming, or virtual meets for people geographically far apart. They provide escapism for the mundane world for some and enable one to experience something they couldn't have, it can be summarized as a dream come to true to reality.
}

\subsection{Put \textit{your} creative hat and suggest what kind of big challenge(s) can be tackled by \textit{PES University} in the near future (2023/2024) by each of these tech?}

\begin{center}
    \textbf{\underline{AI/ML}}
\end{center}

AI/ML could be used to solve multiple challenges in PES University, one of them being, a student helper. This bot or program can analyze what a student is lacking in terms of academic ability and suggest areas of improvement, and this can be made personalized by letting the bot analyze and construct a profile of the student.
\\Since the digitization of answer scripts has already been made possible with the existing use of technology, another layer could be added to analyze the answers written by the users to enable insights and provide resources for improvement.
\\Another area where AI/ML could have a possible application could be to analyze the past trends of the companies visiting for placements and internships and providing an estimate of the details considering the current market and other factors to help students plan their placements better and apply with ease.

\begin{center}
    \textbf{\underline{AR/VR}}
\end{center}

AR/VR could be put to wonderful use to help understand abstract concepts better, such as 3-dimensional problems or understanding the structure of any object. This could especially be put to use for the mechanical branch to help visualize constructions better.
\\They could also provide help in research in areas related to biology to help analyze the data better. Replicate real life simulations into the virtual world.

\subsection{Can \textit{you} think of creative ways of developing this technology \textit{in India} to create impact (financial, societal, etc)? Capture this in 100-200 words. }

\begin{center}
    \textbf{\underline{AI/ML}}
\end{center}

While I'm not an expert in this field by any means, and with it already being put to use, I can only suggest to an extent.
\\AI and ML are very powerful tools that can be utilized to achieve a lot of things. The regular predictive models such as weather could be enhanced and data could be processed faster than a human ever could.
\\Provide insights in market trading and investment. AI/ML could be used to help the curious minds of students with their studies. With the already existing use of this tech in various fields, the fields could be combined to create some more interesting areas of research. Such as Computer Vision and AI to help self-driving cars thrive in the Indian environment, Tesla already accomplished this in the US but I have doubts whether it can survive on the Indian roads.
\\The only way this tech can develop and mature in India would be to spread the knowledge about the workings of the artificial mind. When more people are educated enough to know the workings, they can make significant contributions in this field, which will enable them to receive funding to explore more and cause change.

\begin{center}
    \textbf{\underline{AR/VR}}
\end{center}

The main usage of AR/VR as of now can mainly be seen in gaming. Having the ability to explore another world at the comfort of your home is definitely an opportunity to not miss.
\\Gaming is a huge ocean where everyone wants to dip their hands in, as it provides a large opportunity to create something that many people would like to experience. Racing simulations or experiencing another world via VR is something humans have been looking forward to.
\\Another major sector where AR/VR can be incorporated would be in the education sector. Visualizing things most of the times makes the problem easier to understand, and giving students the ability to visualize will certainly aid them in their development.
\\Other major area of incorporation could be healthcare where the understanding of the human body could be deepened with the help of the visualization that AR/VR could provide.

\rule{\textwidth}{0.4pt}

\newpage

\bibliographystyle{acm}
\bibliography{references}

\end{document}
